\label{sec:optimize}
Following the standard setting of minimizer optimization, we fix $w,k$ for the masked minimizer scheme and optimize for $\pi,\nu$. We address this objective via a bi-level optimization routine, which first optimizes for $\pi$, then performs greedy search for the optimal sub-sampling mask $\nu$, which starts with the minimizer mask $\mathbf{1}_w$ and iteratively zeroes out a position that yields the best performance improvement. The search terminates when no further improvement can be obtained. In principle, any existing optimizer for minimizer \cite{ekim20pasha,zheng20miniception,zheng21,hoang2022deepminimizer} can be used in the first step. However, we note that all existing algorithms to optimize minimizers do not account for the conservation component that is reflected in the GSS metric. To address this, we adapt the \textsc{DeepMinimizer} loss function~\cite{hoang2022deepminimizer} 
with a secondary objective that minimizes the expected change in $k$-mer score assignment with respect to random mutations, hence encouraging high conservation. Our adapted loss function to optimize GSS is given as follows:
\begin{eqnarray}
\mathcal{L}_{gss}(S; \alpha, \beta) \ &\triangleq& \ \Delta(\mathbf{f}(S;\alpha), \mathbf{g}(S;\beta)) + \sum_{i=1}^n \Delta(\mathbf{f}(S_i; \alpha), \mathbf{g}(S; \beta)) \ ,
\label{eq:loss}
\end{eqnarray}
where $S_1, S_2, \dots , S_n$ denote $n$ randomly sampled mutations of $S$. The first term on the right hand side is exactly the density optimizing \textsc{DeepMinimizer} loss~\cite{hoang2022deepminimizer}. The second term minimizes the expected distance between each priority vector $\mathbf{f}(S_i; \alpha)$ (of the mutated sequence $S_i$) and the template vector $\mathbf{g}(S; \beta)$ (of the original sequence). Intuitively, when this term is small, we expect all $\mathbf{f}(S_i; \alpha)$ to be concentrated around $\mathbf{g}(S; \beta)$. As $\mathbf{g}(S; \alpha)$ is close to $\mathbf{f}(S; \alpha)$ via minimizing the first term, this optimality also implies that the score assignment $\mathbf{f}(S; \alpha)$ is likely to be conserved under random mutations. 